\documentclass{article}
\usepackage{xeCJK}
\usepackage{amsmath}
\usepackage{semantic}
\setCJKmainfont{更纱黑体 SC}
\setCJKmonofont{等距更纱黑体 SC}

\author{emliunix}
\title{Lambda, Haskel, etc}
\date{2020/01/06}

\begin{document}

\maketitle

\section*{最近}

好久没写东西了,也好久没用中文了。最近用Haskell写了些东西,还是半成的,但是也积累了些实践经验。
其中有些不清楚的地方,为了更加明确,又看了些基本的$\lambda$演算,以及类型系统。
理解了些东西,又有了很多不理解的东西。

\section*{Typeclass}

首先,赞美Typeclass。用过OCaml的Functor,Haskell的Typeclass,更不用说OO的对象系统了。
总体感觉Typeclass是最趁手的。简短的来说,Typeclass提供了一种轻量又灵活的抽象方式。
Haskell中最为出名的Monad(以及Functor,Applicative,etc)都是以Typeclass定义的。
它也衍生出Scala的implicits,Rust的trait。
Ad-hoc polymorphism是很常见的一种需求,OO的话,就需要定义接口,
实现(实现在类型定义侧,无法独立出来,不够灵活),也没有\emph{自动实现},约束(Constraint)的能力。
而且,OO(Java)总有种涛涛废话东流水的感觉。

对于\emph{自动实现},我指Haskell中存在这些定义:{\tt instance F a => F (List a) where }或者{\tt instance F a => G a where}。
它的作用范围是所有符合要求的类型。而OO中,尚未见识过类似于这种作用于接口上的接口定义。

Functor的问题在于显式,因此失去了隐式带来的灵活性。在使用侧,模块需要一步步构造出来。代码量就多了,
而且代码因为被固定于具体的类型,失去了通用性。

所以Haskell的代码相比就足够抽象,因此足够通用,于是足够灵活,能更多的与其他代码组合。

\section*{语言的内容}

到底什么组成了一个编程语言。比如Typeclass是怎么工作的呢。

所有人都知道$\lambda$演算,也说他是Haskell,OCaml等等常见函数式编程语言的基础,
但是很明显$\lambda$演算和Haskell怕不是隔着好几百个Java。GADTs,Typeclass怎么跑到$\lambda$的。

简单来说Haskell代码编译到Core(System F$\omega$)之后就已经没有GADTs,Typeclass这些了。所以,
至少这两个部分是在Typechecker,Typeinferer这里做的,转换到了Core,并且丢失了一部分信息,
或者说特化了。这种从Surface Language到Core Language的转换过程,如果我没理解错的话,
一般称之为elaborate。

不过这不影响他们是Haskell本身的一个语言(Surface Language)特性,所以从语言模型上来讲,也有着他们的rules。
$\lambda$ cube定义了三种对简单类型的扩展以及这些扩展的组合,但是这只是在类型系统上对语言的扩展。
如ADT构造,if-else,pattern matching这些结构上的扩展也是扩展。

实用的编程语言都是复杂的,有各种各样的结构来提供便利,同时我们又希望他的基础是足够通用与简单的,
这样一方面减轻使用者的负担,另一方面利于发展。这件事情是很难的。

\end{document}